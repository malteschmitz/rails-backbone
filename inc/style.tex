% common packages

\usepackage[utf8]{inputenc}
\usepackage[T1]{fontenc}

\usepackage{lmodern}
\usepackage{textcomp}
\usepackage{courier}

\usepackage[ngerman]{babel}
\usepackage[german=guillemets]{csquotes}

\usepackage{amsmath}
\usepackage{stmaryrd}
\usepackage{tikz}
\usetikzlibrary{positioning,%
  shapes.geometric,%
  shapes.misc,%
  calc,%
  fit,%
  decorations,%
  decorations.pathreplacing,%
  backgrounds,%
  arrows,%
  automata%
}

% define colors used by presentation theme AND article
\xdefinecolor{bordeaux}{RGB}{128,0,50}
\xdefinecolor{niceblue}{RGB}{13,41,79}
\xdefinecolor{nicegreen}{RGB}{13,79,18}
\xdefinecolor{niceviolet}{RGB}{79,13,74}
\colorlet{maincolor}{niceviolet}
\colorlet{alertedcolor}{red}
\colorlet{examplecolor}{green!50!black}
\usetheme{Malte}

% color system
% - color!9 used as light fill color (e.g. for block content)
% - color!18 used as default fill color (e.g. for state)
% - color!30 used as highlight fill color (e.g. for block header)
% - color!50 used as strong highlight fill color (e.g. for header row)

\usepackage{listings}
\lstset{%
  basicstyle=\ttfamily,%
  keywordstyle=\color{maincolor}\bfseries,%
  identifierstyle=\color{blue},%
  numbers=none,%
  frame=lines,%
  captionpos={b},% set caption pos to bottom
  backgroundcolor=\color{maincolor!10},%
  rulecolor=\color{maincolor!70},%
  framerule=1pt,%
  showstringspaces=false,%
  upquote=true}
  
% zebra tables
\newcommand{\mainrowcolors}{\rowcolors{1}{maincolor!30}{maincolor!9}}
\newenvironment{zebratabular}{\mainrowcolors\begin{tabular}}{\end{tabular}}
\newcommand{\setrownumber}[1]{\global\rownum#1\relax}
\newcommand{\headerrow}{\rowcolor{maincolor!50}\setrownumber1}

\AtBeginDocument{\hypersetup{
  pdftitle={\titlestring},
  pdfauthor={\authorstring}}
  \title[\shorttitlestring]{\titlestring}
  \author{\authorstring}
  \date{\datestring}}

% set texts used later
\newcommand{\targetsname}{Ziele dieses Vortrags}
\newcommand{\targetscontentname}{Ziele und Inhalt}
\newcommand{\outlinename}{Gliederung}

\iflanguage{english}{
  \renewcommand{\targetsname}{Targets of this Talk}
  \renewcommand{\targetscontentname}{Targets \& Outline}
  \renewcommand{\outlinename}{Outline}
}

% often used sets
\newcommand{\set}[1]{\mathbb{#1}}
\newcommand{\R}{\set{R}}
\newcommand{\N}{\set{N}}
\newcommand{\Z}{\set{Z}}
\newcommand{\Q}{\set{Q}}

% easier use of operatorname
\newcommand{\op}[1]{\operatorname{#1}}

% display vectors in bold
\let\oldvec\vec
\let\vec\boldsymbol

\newcommand{\xskip}{\only<presentation>{\vskip1em}}



% set options for article
\mode
<article>

\newcommand{\itemref}[1]{#1}
\newcommand{\term}[1]{#1}

\KOMAoption{fontsize}{10pt}
\KOMAoption{parskip}{half}
\KOMAoption{headings}{normal}
\KOMAoption{numbers}{noendperiod}

\usepackage[paper=a4paper, left=3cm, right=4cm, top=3cm, bottom=4cm, headsep=1cm, marginparsep=0em, marginparwidth=3em]{geometry}

\usepackage{paralist}
% use compactitem as default itemize environment
\newenvironment{looseitemize}{}{}
\let\looseitemize\itemize
\let\endlooseitemize\enditemize
\let\itemize\compactitem
\let\enditemize\endcompactitem
% use compactenum as default enumerate environment
\newenvironment{looseenumerate}{}{}
\let\looseenumerate\enumerate
\let\endlooseenumerate\endenumerate
\let\enumerate\compactenum
\let\endenumerate\endcompactenum

% create head and foot line
\usepackage{scrpage2}
\pagestyle{scrheadings}

% stretch headline above margin (scrguide -- page 244)
\setheadwidth[0pt]{textwithmarginpar}

\usepackage{tocstyle}
\usetocstyle[allwithdot]{standard}
\makeatletter
\newcommand{\targets}[1]{
  \parbox[t]{0.48\textwidth}{\raggedright
    \paragraph*{\targetsname}\ \\[2ex]
    \begin{enumerate}#1\end{enumerate}
  }
  \hskip0.04\textwidth
  \parbox[t]{0.48\textwidth}{\raggedright
    \paragraph*{\contentsname}\ \\[2ex]
    \@starttoc{toc}
  }
  % emulate two frames
  \frame{}
  \frame{}
}
\makeatother


\clearscrheadfoot % clear all default settings
\automark{section} % \automark[\rightmark]{\leftmark}

\ihead{\shorttitlestring\ \textcolor{maincolor}{\bfseries |} \leftmark \hfill \pagemark}
\setkomafont{pageheadfoot}{\normalfont\sffamily}
\setkomafont{pagenumber}{\usekomafont{pageheadfoot}\color{maincolor}}
\setheadsepline{0.5pt}[\color{maincolor}]




% hide block title in article mode
\setbeamertemplate{block begin}{}
\setbeamertemplate{block end}{}

% display block title in Block environment
\newenvironment{Block}{%
  \setbeamertemplate{block begin}[default]
  \setbeamertemplate{block end}[default]
  \begin{block}}{\end{block}}

% hide frame title and subtitle in article mode for default frame environment
\setbeamertemplate{frametitle}{}

% display frame title of Frame environment
\newenvironment{Frame}{%
  \setbeamertemplate{frametitle}{%
    \paragraph*{%
      \insertframetitle%
      \ifx\insertframesubtitle\empty\else%
        \newline%
        \mdseries\insertframesubtitle%
      \fi%
    }\ \par}
  \begin{frame}}{\end{frame}}

% hide effect of beameritemize
\newenvironment{beameritemize}{%
  \let\item\relax\ignorespaces}{%
  \ignorespacesafterend}
  
% map beamerenumerate to inparaenum
\newenvironment{beamerenumerate}{%
  \begin{inparaenum}\ignorespaces}{%
  \end{inparaenum}\ignorespacesafterend}

% map plain command to frame
\newcommand{\plain}[1]{\frame{#1}}

% display frame number
\setbeamertemplate{frame end}{%
  \marginpar{\hfill\scriptsize\sffamily\color{maincolor}$\to\,$\insertframenumber}}
      
\usepackage{hyperref}
\hypersetup{breaklinks=true,
            pdfborder={0 0 0},
            pdfhighlight={/N}}
            
\newcommand{\displaytitle}{
  \begin{frame}
    \thispagestyle{plain}
    \begin{center}
      \color{maincolor}\sffamily
      \huge\titlestring
      \par\large
      \authorstring\ -- \datestring
    \end{center}
    \vskip2em
  \end{frame}}

\usepackage{verbatim} % for the comment environment

% Set options for beamer
\mode
<presentation>

\newcommand{\itemref}[1]{\textbf{\color{maincolor}#1}}

\newcommand{\term}[1]{{\color{blue}#1}}

\newcommand{\displaytitle}{
  \plain{
    \begin{center}
      \null\vfill
      \color{maincolor}\sffamily
      \huge\titlestring
      \par\vskip2ex
      \large\authorstring\\\datestring
      \vskip12ex
      \vfill
    \end{center}}}
  
\newcommand{\targets}[1]{\section*{\targetscontentname}
  \frame[plain]{\frametitle{\targetsname}%
  \begin{enumerate}#1\end{enumerate}}
  \frame[plain]{\frametitle{\outlinename}\tableofcontents}}

% create new environment Block behaving exactly like block
\newenvironment{Block}{%
  \begin{block}}{\end{block}}

% create new environment Frame behaving exactly like frame in beamer mode
\newenvironment{Frame}[1][]{%
  \begin{frame}[environment=Frame,#1]}{\end{frame}}

% create new environment beameritemize behaving exactly like itemize
\newenvironment{beameritemize}{%
  \begin{itemize}}{\end{itemize}}

% create new environment beamerenumerate behaving exactly like enumerate
\newenvironment{beamerenumerate}{%
  \begin{enumerate}}{\end{enumerate}}


\mode
<all>

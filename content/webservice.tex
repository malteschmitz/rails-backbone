%\begin{frame}{}
%  \begin{itemize}
%    \item item
%  \end{itemize}
%\end{frame}

\colorlet{maincolor}{bordeaux} % Color section in bordeaux

\section[Webservice]{Webservice mit Rails 3.2}

\begin{frame}{Was ist Rails}
  \begin{itemize}
    \item Ruby on Rails Web-Framework
    \item Annahmen über \enquote{The Rails Way}
    \item Folgt Prinzipien
    \begin{itemize}
      \item DRY -- \enquote{Don’t Repeat Yourself}
      \item Convention Over Configuration
      \item REST-Architekturstil
    \end{itemize}
  \end{itemize}
\end{frame}

%%%%%%%%%%%%%%%%%%%%%%%%%%%%%%%%%%%%%%%%%%%%%%%%%%%%%%%%%%%%%%%%%%%%%%%%%%%%%%%%
\subsection{Model-View-Controller}

\begin{frame}{Models}
  \begin{itemize}
    \item Repräsentieren Daten
    \item Interagieren mit der Datenbank
    \item Enthalten Applikationslogik
    \item Häufig: Model $\hat =$ Datenbank-Tabelle
  \end{itemize}
\end{frame}

\begin{frame}{Views}
  \begin{itemize}
    \item Bieten Benutzungsschnittstelle der Applikation
    \item Präsentieren ausschließlich Daten
    \item Liefern Daten aus
    \item Häufig: HTML-Dokumente mit eingebettetem Ruby-Code
  \end{itemize}
\end{frame}

\begin{frame}{Controllers}
  \begin{itemize}
    \item Integrieren Views und Models
    \item Verarbeiten Anfragen
    \item Rufen Daten beim Model ab
    \item Liefern Daten zum View
  \end{itemize}
\end{frame}

\begin{frame}{Komponentenübersicht}
  \begin{center}
    \begin{tikzpicture}[
    node/.style={
      draw=maincolor,
      thick,
      rectangle,
      rounded corners,
      minimum width=6em,
      minimum height=4ex,
      inner sep=0ex,
      fill=maincolor!18},
    shorten >=1pt,
    xs/.style={every edge/.append style={
      transform canvas={xshift=#1}
    }},
    ys/.style={every edge/.append style={
      transform canvas={yshift=#1}
    }},
    line/.style={
      thick,
      color=maincolor
    },
    node distance=2em and 5ex
  ]

  \node (Client) {\shortstack{\pgfimage[width=2.5em]{images/client}\\ Client}};
  \node[node, node distance=4ex, below=of Client] (WEBrick) {WEBrick};
  \node[node distance=10ex,left=of WEBrick] (Files) {\shortstack{\pgfimage[width=2.5em]{images/db}\\ Files}};
  \node[node, node distance=2em and -5ex, below left=of WEBrick] (Routing) {Routing};
  \node[node, node distance=2em and -5ex, below right=of WEBrick] (Views) {Views};
  \node[node, node distance=2em and -5ex, below right=of Routing] (Controllers) {Controllers};
  \node[node, node distance=4ex, below=of Controllers] (Models) {Models};
  \node[node distance=10ex, left=of Models] (Database) {\shortstack{\pgfimage[width=2.5em]{images/db}\\ Database}};

  \draw[line, xs=-1em]
    (Client) edge[->] (WEBrick)
    (WEBrick) edge[->,out=270,in=90] ($(Routing.north) + (1em,0)$)
    ($(Routing.south) + (1em,0)$) edge[->,out=270,in=90] (Controllers)
    (Controllers) edge[->] (Models);

  \draw[line, xs=1em]
    (Models) edge[->] (Controllers)
    (WEBrick) edge[->] (Client);

  \draw[line, xs=1em]
    (Controllers) edge[->,out=90,in=270] ($(Views.south) + (-1em,0)$)
    ($(Views.north) + (-1em,0)$) edge[->,out=90,in=270] (WEBrick);

  \draw[line, ys=1ex]
    (Database) edge[->] (Models)
    (WEBrick) edge[->] (Files);

  \draw[line, ys=-1ex]
    (Models) edge[->] (Database)
    (Files) edge (WEBrick);
\end{tikzpicture}
  \end{center}
\end{frame}

\begin{frame}{RESTful Rails}
  \begin{itemize}
    \item URIs repräsentieren Ressourcen
    \item Repräsentationen von Ressourcen
    \item Zwei URIs
    \begin{itemize}
      \item Collection-URI \\ \lstinline|GET /users|
      \item Item-URI \\ \lstinline|GET /users/42|
    \end{itemize}
  \end{itemize}
\end{frame}

\begin{frame}{RESTful Rails}
  \begin{center}
    \begin{zebratabular}{lll}\headerrow
      Aktion  & Methode & Bedeutung \\
      index   & GET     & Abrufen aller Entitäten \\
      create  & POST    & Anlegen einer Entität \\
      show    & GET     & Abrufen einer Entität \\
      update  & PUT     & Aktualisieren einer Entität \\
      destroy & DELETE  & Löschen einer Entität \\

      \only<1>{new}\only<2>{\color{red}\xcancel{\color{black}new}} &
      \only<1>{GET}\only<2>{\color{red}\xcancel{\color{black}GET}} &
      Formular zum Erzeugen \\
      \only<1>{edit}\only<2>{\color{red}\xcancel{\color{black}edit}} &
      \only<1>{GET}\only<2>{\color{red}\xcancel{\color{black}GET}} &
      Formular zum Editieren \\
    \end{zebratabular}
  \end{center}
\end{frame}

%%%%%%%%%%%%%%%%%%%%%%%%%%%%%%%%%%%%%%%%%%%%%%%%%%%%%%%%%%%%%%%%%%%%%%%%%%%%%%%%
\subsection{Rails3 Responder}

\begin{frame}[fragile]{Alte Implementierung}
  \begin{lstlisting}[language=Ruby,gobble=4,basicstyle=\ttfamily\small]
    class UsersController < ApplicationController
      def index
        @users = User.all
        respond_to do |format|
          format.html # index.html.erb
          format.xml  { render :xml => @users}
          format.json { render :json => @users}
        end
      end

      ...
    end
  \end{lstlisting}
\end{frame}

\begin{frame}[fragile]{Neue Implementierung}
  \begin{lstlisting}[language=Ruby,gobble=4,basicstyle=\ttfamily\small]
    class UsersController < ApplicationController
      def index
        @users = User.all
        respond_with @users
      end

      ...
    end
  \end{lstlisting}
\end{frame}

%%%%%%%%%%%%%%%%%%%%%%%%%%%%%%%%%%%%%%%%%%%%%%%%%%%%%%%%%%%%%%%%%%%%%%%%%%%%%%%%
\subsection{Views}

\frame{no content}

%%%%%%%%%%%%%%%%%%%%%%%%%%%%%%%%%%%%%%%%%%%%%%%%%%%%%%%%%%%%%%%%%%%%%%%%%%%%%%%%
\subsection{Assoziationen}

\frame{no content}

%%%%%%%%%%%%%%%%%%%%%%%%%%%%%%%%%%%%%%%%%%%%%%%%%%%%%%%%%%%%%%%%%%%%%%%%%%%%%%%%
\subsection{Validierung}

\frame{no content}



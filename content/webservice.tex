%\begin{frame}{}
%  \begin{itemize}
%    \item item
%  \end{itemize}
%\end{frame}

\colorlet{maincolor}{bordeaux} % Color section in bordeaux

\section[Webservice]{Webservice mit Rails 3.2}

\begin{frame}{Was ist Rails}
  \begin{itemize}
    \item Ruby on Rails Web-Framework
    \item Annahmen über \enquote{The Rails Way}
    \item Folgt Prinzipien
    \begin{itemize}
      \item DRY -- \enquote{Don’t Repeat Yourself}
      \item Convention Over Configuration
      \item REST-Architekturstil
    \end{itemize}
  \end{itemize}
\end{frame}

%%%%%%%%%%%%%%%%%%%%%%%%%%%%%%%%%%%%%%%%%%%%%%%%%%%%%%%%%%%%%%%%%%%%%%%%%%%%%%%%
\subsection{Model-View-Controller}

\begin{frame}{Models}
  \begin{itemize}
    \item Repräsentieren Daten
    \item Interagieren mit der Datenbank
    \item Enthalten Applikationslogik
    \item Häufig: Model $\hat =$ Datenbank-Tabelle
  \end{itemize}
\end{frame}

\begin{frame}{Views}
  \begin{itemize}
    \item Bieten Benutzungsschnittstelle der Applikation
    \item Präsentieren ausschließlich Daten
    \item Liefern Daten aus
    \item Häufig: HTML-Dokumente mit eingebettetem Ruby-Code
  \end{itemize}
\end{frame}

\begin{frame}{Controllers}
  \begin{itemize}
    \item Integrieren Views und Models
    \item Verarbeiten Anfragen
    \item Rufen Daten beim Model ab
    \item Liefern Daten zum View
  \end{itemize}
\end{frame}

%%%%%%%%%%%%%%%%%%%%%%%%%%%%%%%%%%%%%%%%%%%%%%%%%%%%%%%%%%%%%%%%%%%%%%%%%%%%%%%%
\subsection{REST mit Rails}

\frame{no content}

%%%%%%%%%%%%%%%%%%%%%%%%%%%%%%%%%%%%%%%%%%%%%%%%%%%%%%%%%%%%%%%%%%%%%%%%%%%%%%%%
\subsection{Responder aus Rails 3}

\frame{no content}

%%%%%%%%%%%%%%%%%%%%%%%%%%%%%%%%%%%%%%%%%%%%%%%%%%%%%%%%%%%%%%%%%%%%%%%%%%%%%%%%
\subsection{Views}

\frame{no content}

%%%%%%%%%%%%%%%%%%%%%%%%%%%%%%%%%%%%%%%%%%%%%%%%%%%%%%%%%%%%%%%%%%%%%%%%%%%%%%%%
\subsection{Assoziationen}

\frame{no content}

%%%%%%%%%%%%%%%%%%%%%%%%%%%%%%%%%%%%%%%%%%%%%%%%%%%%%%%%%%%%%%%%%%%%%%%%%%%%%%%%
\subsection{Validierung}

\frame{no content}



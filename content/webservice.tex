%\begin{frame}{}
%  \begin{itemize}
%    \item item
%  \end{itemize}
%\end{frame}

\colorlet{maincolor}{bordeaux} % Color section in bordeaux
\lstset{
  language=Ruby,
  gobble=4,
  basicstyle=\ttfamily\small
}

\section[Webservice]{Webservice mit Rails 3.2}

\begin{frame}{Was ist Rails}
  \begin{itemize}
    \item Ruby on Rails Web-Framework
    \item Annahmen über \enquote{The Rails Way}
    \item Folgt Prinzipien
    \begin{itemize}
      \item DRY -- \enquote{Don’t Repeat Yourself}
      \item Convention Over Configuration
      \item REST-Architekturstil
    \end{itemize}
  \end{itemize}
\end{frame}

%%%%%%%%%%%%%%%%%%%%%%%%%%%%%%%%%%%%%%%%%%%%%%%%%%%%%%%%%%%%%%%%%%%%%%%%%%%%%%%%
\subsection{Model-View-Controller}

\begin{frame}[fragile]{Model-View-Controller}
  \begin{center}
    \begin{tikzpicture}[
    node/.style={
      draw=maincolor,
      thick,
      rectangle,
      rounded corners,
      minimum width=6em,
      minimum height=7ex,
      inner sep=0ex,
      fill=maincolor!18},
    shorten >=1pt,
    xs/.style={every edge/.append style={
      transform canvas={xshift=#1}
    }},
    ys/.style={every edge/.append style={
      transform canvas={yshift=#1}
    }},
    line/.style={
      thick,
      color=maincolor
    },
    node distance=2em and 0ex
  ]

  \node[node] (Controller) {Controller};
  \node[node, below left=of Controller] (View) {View};
  \node[node, below right=of Controller] (Model) {Model};


  \draw[line]
    (Controller) edge [->] (View)
    (Controller) edge [->] (Model);
    
  \draw[line, ys=-1ex]
    (View) edge [->] (Model);
  
  \draw[line, ys=1ex,dashed]
    (Model) edge [->] (View);
    
  \draw[line,dashed]
    (View) edge [->, out=90, in=180] (Controller);

\end{tikzpicture}

  \end{center}
\end{frame}

\begin{frame}{Models}
  \begin{itemize}
    \item Repräsentieren Daten
    \item Interagieren mit der Datenbank
    \item Enthalten Applikationslogik
    \item Häufig: Model $\hat =$ Datenbank-Tabelle
  \end{itemize}
\end{frame}

\begin{frame}{Views}
  \begin{itemize}
    \item Bieten Benutzungsschnittstelle der Applikation
    \item Präsentieren ausschließlich Daten
    \item Liefern Daten aus
    \item Häufig: HTML-Dokumente mit eingebettetem Ruby-Code
  \end{itemize}
\end{frame}

\begin{frame}{Controllers}
  \begin{itemize}
    \item Integrieren Views und Models
    \item Verarbeiten Anfragen
    \item Rufen Daten beim Model ab
    \item Liefern Daten zum View
  \end{itemize}
\end{frame}

\begin{frame}{Komponentenübersicht}
  \begin{center}
    \begin{tikzpicture}[
    node/.style={
      draw=maincolor,
      thick,
      rectangle,
      rounded corners,
      minimum width=6em,
      minimum height=4ex,
      inner sep=0ex,
      fill=maincolor!18},
    shorten >=1pt,
    xs/.style={every edge/.append style={
      transform canvas={xshift=#1}
    }},
    ys/.style={every edge/.append style={
      transform canvas={yshift=#1}
    }},
    line/.style={
      thick,
      color=maincolor
    },
    node distance=2em and 5ex
  ]

  \node (Client) {\shortstack{\pgfimage[width=2.5em]{images/client}\\ Client}};
  \node[node, node distance=4ex, below=of Client] (WEBrick) {WEBrick};
  \node[node distance=10ex,left=of WEBrick] (Files) {\shortstack{\pgfimage[width=2.5em]{images/db}\\ Files}};
  \node[node, node distance=2em and -5ex, below left=of WEBrick] (Routing) {Routing};
  \node[node, node distance=2em and -5ex, below right=of WEBrick] (Views) {Views};
  \node[node, node distance=2em and -5ex, below right=of Routing] (Controllers) {Controllers};
  \node[node, node distance=4ex, below=of Controllers] (Models) {Models};
  \node[node distance=10ex, left=of Models] (Database) {\shortstack{\pgfimage[width=2.5em]{images/db}\\ Database}};

  \draw[line, xs=-1em]
    (Client) edge[->] (WEBrick)
    (WEBrick) edge[->,out=270,in=90] ($(Routing.north) + (1em,0)$)
    ($(Routing.south) + (1em,0)$) edge[->,out=270,in=90] (Controllers)
    (Controllers) edge[->] (Models);

  \draw[line, xs=1em]
    (Models) edge[->] (Controllers)
    (WEBrick) edge[->] (Client);

  \draw[line, xs=1em]
    (Controllers) edge[->,out=90,in=270] ($(Views.south) + (-1em,0)$)
    ($(Views.north) + (-1em,0)$) edge[->,out=90,in=270] (WEBrick);

  \draw[line, ys=1ex]
    (Database) edge[->] (Models)
    (WEBrick) edge[->] (Files);

  \draw[line, ys=-1ex]
    (Models) edge[->] (Database)
    (Files) edge (WEBrick);
\end{tikzpicture}
  \end{center}
\end{frame}

\begin{frame}{RESTful Rails}
  \begin{itemize}
    \item Addressierung von Ressourcen durch URIs
    \item Austausch von Zustands-Repräsentationen
    \item Zwei URIs
    \begin{itemize}
      \item Collection-URI \\ \lstinline|GET /users|
      \item Item-URI \\ \lstinline|GET /users/42|
    \end{itemize}
  \end{itemize}
\end{frame}

\begin{frame}[fragile]{RESTful Rails}
  \newcommand{\coord}[1]{\tikz[remember picture]{\node[coordinate](#1){};}}

  \begin{center}
    \begin{zebratabular}{lll}\headerrow
      Aktion  & Methode & Bedeutung \\
      index   & GET     & Abrufen aller Entitäten \\
      create  & POST    & Anlegen einer Entität \\
      show    & GET     & Abrufen einer Entität \\
      update  & PUT     & Aktualisieren einer Entität \\
      destroy & DELETE  & Löschen einer Entität \\
      \coord{new left}new & GET & Formular zum Erzeugen\coord{new right}\\
      \coord{edit left}edit & GET & Formular zum Editieren\coord{edit right}
    \end{zebratabular}
  \end{center}

  \begin{tikzpicture}[remember picture,overlay]
    \only<2>{\draw[red,thick]
      ($(new left) + (-.5ex,.5ex) $) --
      ($(new right) + (.5ex,.5ex)$)
      ($(edit left) + (-.5ex,.5ex) $) --
      ($(edit right) + (.5ex,.5ex)$);}
  \end{tikzpicture}
\end{frame}

%%%%%%%%%%%%%%%%%%%%%%%%%%%%%%%%%%%%%%%%%%%%%%%%%%%%%%%%%%%%%%%%%%%%%%%%%%%%%%%%
\subsection{Responder}

\begin{frame}[fragile]{Responder}
  Alte Implementierung:
  \begin{lstlisting}
    class PlanetsController < ApplicationController
      def index
        @planets = Planet.all
        respond_to do |format|
          format.html # index.html.erb
          format.xml  { render :xml => @planets}
          format.json { render :json => @planets}
        end
      end

      ...
    end
  \end{lstlisting}
\end{frame}

\begin{frame}[fragile]{Responder}
  Neue Implementierung:
  \begin{lstlisting}
    class PlanetsController < ApplicationController
      respond_to :html, :xml, :json

      def index
        @planets = Planet.all
        respond_with(@planets)
      end

      ...
    end
  \end{lstlisting}
\end{frame}

\begin{frame}{Responder Ablauf}
  Request for an XML response:
  \begin{enumerate}
    \item search for a template in \lstinline|people/index.xml|
    \item invoke \lstinline|to_xml|
    \item call \lstinline|to_format|
  \end{enumerate}
\end{frame}

\begin{frame}[fragile]{HTTP-Semantik}
  \begin{lstlisting}
    def create
      @planet = Planet.new(params[:planet])
      if @planet.save
        flash[:notice] = 'Planet created.'
      end
      respond_with(@planet)
    end
  \end{lstlisting}

  \begin{description}
    \item[Erfolg] Status 201 Created
    \item[Fehler] Status 422 Unprocessable Entity
  \end{description}
\end{frame}

%%%%%%%%%%%%%%%%%%%%%%%%%%%%%%%%%%%%%%%%%%%%%%%%%%%%%%%%%%%%%%%%%%%%%%%%%%%%%%%%
\subsection{Views}

\begin{frame}[fragile]{ERB Views}
  \begin{itemize}
    \item Ruby Code eingebettet in HTML-Dokumente
    \item Unterstützt Partials
  \end{itemize}

  \begin{lstlisting}[language=HTML]
    <h1>Sonnensystem <%=h @solarsystem %></h1>
    <% @planets.each do |planet| %>
      <%= render @planet %>
    <% end %>
  \end{lstlisting}

  \begin{lstlisting}[language=HTML]
    <p>
      Der Planet <%=h planet.name =>
      hat die Masse <%=h planet.mass =>
      und die Albedo <%=h planet.albedo =>.
    </p>
  \end{lstlisting}
\end{frame}

\begin{frame}[fragile]{JSON- und XML-Repräsentationen}
  \lstset{identifierstyle=\color{black}}

  \begin{columns}[t]
    \begin{column}{.45\textwidth}
      \begin{lstlisting}[language={},gobble=8]
        {
          "id": 42,
          "name": "Mars",
          "mass": 6.419e23,
          "albedo": 0.15
        }
      \end{lstlisting}
    \end{column}
    \begin{column}{.45\textwidth}
      \begin{lstlisting}[language=XML,gobble=8]
        <planet
          id="42"
          name="Mars"
          mass="6.419e23"
          albedo="0.15"
        />
      \end{lstlisting}
    \end{column}
  \end{columns}
\end{frame}

%%%%%%%%%%%%%%%%%%%%%%%%%%%%%%%%%%%%%%%%%%%%%%%%%%%%%%%%%%%%%%%%%%%%%%%%%%%%%%%%
\subsection{Assoziationen}

\frame{no content}

%%%%%%%%%%%%%%%%%%%%%%%%%%%%%%%%%%%%%%%%%%%%%%%%%%%%%%%%%%%%%%%%%%%%%%%%%%%%%%%%
\subsection{Validierung}

\frame{no content}



%\begin{frame}{}
%  \begin{itemize}
%    \item item
%  \end{itemize}
%\end{frame}

\colorlet{maincolor}{bordeaux} % Color section in bordeaux

\section[Webservice]{Webservice mit Rails 3.2}

\begin{frame}{Was ist Rails}
  \begin{itemize}
    \item Ruby on Rails Web-Framework
    \item Annahmen über \enquote{The Rails Way}
    \item Folgt Prinzipien
    \begin{itemize}
      \item DRY -- \enquote{Don’t Repeat Yourself}
      \item Convention Over Configuration
      \item REST-Architekturstil
    \end{itemize}
  \end{itemize}
\end{frame}

%%%%%%%%%%%%%%%%%%%%%%%%%%%%%%%%%%%%%%%%%%%%%%%%%%%%%%%%%%%%%%%%%%%%%%%%%%%%%%%%
\subsection{Model-View-Controller}

\begin{frame}{Models}
  \begin{itemize}
    \item Repräsentieren Daten
    \item Interagieren mit der Datenbank
    \item Enthalten Applikationslogik
    \item Häufig: Model $\hat =$ Datenbank-Tabelle
  \end{itemize}
\end{frame}

\begin{frame}{Views}
  \begin{itemize}
    \item Bieten Benutzungsschnittstelle der Applikation
    \item Präsentieren ausschließlich Daten
    \item Liefern Daten aus
    \item Häufig: HTML-Dokumente mit eingebettetem Ruby-Code
  \end{itemize}
\end{frame}

\begin{frame}{Controllers}
  \begin{itemize}
    \item Integrieren Views und Models
    \item Verarbeiten Anfragen
    \item Rufen Daten beim Model ab
    \item Liefern Daten zum View
  \end{itemize}
\end{frame}

\begin{frame}{Komponentenübersicht}
  \begin{center}
    \begin{tikzpicture}[
      node/.style={draw,rectangle,rounded corners,
                         minimum width=5em,minimum height=3ex,inner sep=0ex,
                         fill=maincolor!18},
                   shorten >=1pt]

      \node (client) {Client};
      \node[node,below of= client] (webrick) {WEBrick};
      \node[node,left of= webrick,node distance=7em] (files) {Files};
      \node[node,below of= webrick] (routing) {Routing};
      \node[node,right of= routing,node distance=7em] (views) {Views};
      \node[node,below of= routing] (controllers) {Controllers};
      \node[node,below of= controllers] (models) {Models};
      \node[node,below of= models] (database) {Database};

      \draw[->,thick,color=maincolor,
            every edge/.append style={transform canvas={xshift=-1em}}]
        (client) edge (webrick)
        (webrick) edge (routing)
        (routing) edge (controllers)
        (controllers) edge (models)
        (models) edge (database);

      \draw[->,thick,color=maincolor,
            every edge/.append style={transform canvas={xshift=1em}}]
        (database) edge (models)
        (models) edge (controllers)
        (webrick) edge (client);

      \draw[thick,color=maincolor]
        (controllers.east) edge[->,out=0,in=-90] (views.south)
        (views.north) edge[->,out=90,in=0] (webrick.east);

      \draw[->,thick,color=maincolor,
            every edge/.append style={transform canvas={yshift=.6ex}}]
        (webrick) edge (files);

      \draw[->,thick,color=maincolor,
            every edge/.append style={transform canvas={yshift=-.6ex}}]
        (files) edge (webrick);
    \end{tikzpicture}
  \end{center}
\end{frame}

%%%%%%%%%%%%%%%%%%%%%%%%%%%%%%%%%%%%%%%%%%%%%%%%%%%%%%%%%%%%%%%%%%%%%%%%%%%%%%%%
\subsection{REST mit Rails}

\frame{no content}

%%%%%%%%%%%%%%%%%%%%%%%%%%%%%%%%%%%%%%%%%%%%%%%%%%%%%%%%%%%%%%%%%%%%%%%%%%%%%%%%
\subsection{Responder aus Rails 3}

\frame{no content}

%%%%%%%%%%%%%%%%%%%%%%%%%%%%%%%%%%%%%%%%%%%%%%%%%%%%%%%%%%%%%%%%%%%%%%%%%%%%%%%%
\subsection{Views}

\frame{no content}

%%%%%%%%%%%%%%%%%%%%%%%%%%%%%%%%%%%%%%%%%%%%%%%%%%%%%%%%%%%%%%%%%%%%%%%%%%%%%%%%
\subsection{Assoziationen}

\frame{no content}

%%%%%%%%%%%%%%%%%%%%%%%%%%%%%%%%%%%%%%%%%%%%%%%%%%%%%%%%%%%%%%%%%%%%%%%%%%%%%%%%
\subsection{Validierung}

\frame{no content}



\colorlet{maincolor}{nicegreen} % Color section in green
\section{Webanwendungen}

\begin{frame}{Webanwendungen vs. Desktopanwendung}
  \begin{center}
    \begin{tikzpicture}[
    box/.style={
      anchor=center,
      rotate=90,
      minimum width=#1,
      minimum height=5ex,
      inner sep=0pt,
      font=\small,
      draw=maincolor,
      thick,
      fill=maincolor!30,
      rounded corners
    },
    app/.style={
      anchor=center,
      rotate=90,
      inner sep=0pt,
      font=\large
    }
  ]
  
  \fill[maincolor!18]
    (0,5ex) rectangle (25em,15ex)
    (0,25ex) rectangle (25em,35ex);
  \fill[maincolor!9]
    (0,-5ex) rectangle (25em,5ex)
    (0,15ex) rectangle (25em,25ex);
  
  \begin{scope}[anchor=west,xshift=1em,font=\large]
    \node at (0,30ex) (Darstellung) {Darstellung};
    \node at (0,20ex) (Präsentation) {Präsentation};
    \node at (0,10ex) (Geschäftslogik) {Geschäftslogik};
    \node at (0,0) (Datenhalung) {Datenhaltung};
  \end{scope}
  
  \begin{scope}[xshift=10.5em]  
    \node at (0em,15ex) [app] {Desktop Application};
    \node at (2em,20ex) [box=28ex] {Client};
    \node at (2em,0ex) [box=8ex] {Server};
    
    \node at (5em,15ex) [app] {Web Application};
    \node at (7em,30ex) [box=8ex] {Client};
    \node at (7em,10ex) [box=28ex] {Server};
    
    \node at (10em,15ex) [app] {Single-Page-Application};
    \node at (12em,25ex) [box=18ex] {Client};
    \node at (12em,5ex) [box=18ex] {Server};
  \end{scope}  
\end{tikzpicture}
  \end{center}
\end{frame}

%%%%%%%%%%%%%%%%%%%%%%%%%%%%%%%%%%%%%%%%%%%%%%%%%%%%%%%%%%%%%%%%%%%%%%%%%%%%%%%%
\subsection{moderne Webanwendungen}

\begin{Frame}{moderne Webanwendungen}{}
  \begin{columns}
    \column{2cm}
    \column{\textwidth}
    {\color<presentation>{maincolor}\bfseries Geschäftslogik im Server} \newline
      \hspace*{1em} zentrale Berechnung und Datenhaltung
    
    \vskip1ex
    
    {\color<presentation>{maincolor}\bfseries zustandsloser Server} \newline
      \hspace*{1em} Abfragen in sich komplett
    
    \vskip1ex
    
    {\color<presentation>{maincolor}\bfseries Daten auf der Leitung} \newline
      \hspace*{1em} Daten frei von Darstellung
    
    \vskip1ex
      
    {\color<presentation>{maincolor}\bfseries Präsentation im Client} \newline
      \hspace*{1em} spezielle Clients möglich
  \end{columns}
\end{Frame}

%%%%%%%%%%%%%%%%%%%%%%%%%%%%%%%%%%%%%%%%%%%%%%%%%%%%%%%%%%%%%%%%%%%%%%%%%%%%%%%%
\subsection{RESTful Webservice}

\begin{Frame}{Was sind Webservices}
  \begin{itemize}
    \item Stellen Funktionalitäten über ein Netzwerk bereit
    \item Eindeutig definierte Schnittstelle
    \item Kommunikation über definierte Protokolle
    \item Zwei bekannte Vertreter: SOAP und RESTful
  \end{itemize}
\end{Frame}

\begin{Frame}{Representational State Transfer (REST)}
  \begin{tabularx}{\textwidth}{rX}
    \textbf{\color<presentation>{maincolor}Identität} & Ressource eindeutig identifiziert \\
    \textbf{\color<presentation>{maincolor}Manipulation} & Selbst beschreibende Ressourcen \\
    \textbf{\color<presentation>{maincolor}Nachrichten} & Alle Informationen 
      zur Verarbeitung \newline
      enthalten\\
    \textbf{\color<presentation>{maincolor}Hypermedia} & 
      \textcolor<presentation>{maincolor}{--} Wenige Einstiegspunkte \newline
      \textcolor<presentation>{maincolor}{--} Interaktion durch Hypermedia \\
    \textbf{\color<presentation>{maincolor}Zustandslos} & Status einer Interaktion mit dem
      Client unbekannt \\
  \end{tabularx}
\end{Frame}

\begin{Frame}{Was sind RESTful Webservices}
  \begin{itemize}
    \item Representational State Transfer (REST)
    \item Bekannter Vertreter WWW
    \item HTTP als Kommunikationsprotokoll
    \item Ressourcen als Elementen der Domäne
    \item Nutzt Verben des HTTP-Protokolls
    \begin{description}
      \item[GET] zum Informationsabruf
      \item[POST] zum Anlegen neuer Daten
      \item[PUT] zum Aktualisieren 
      \item[DELETE] zum Löschen von Daten
    \end{description}
  \end{itemize}
\end{Frame}

%%%%%%%%%%%%%%%%%%%%%%%%%%%%%%%%%%%%%%%%%%%%%%%%%%%%%%%%%%%%%%%%%%%%%%%%%%%%%%%%
\subsection{Single-Page-Application}

\begin{frame}{Single-Page-Application}
  \begin{itemize}
    \item gesamter Client durch Laden einer Website im Browser
    \item Daten werden per AJAX bei Bedarf nachgeladen\\[2ex]
    \item[\goodmark] Client agiert als Anwendung\only<article>{:
      Navigationselemente werden nicht neu geladen und Flackern der Seite wird
      vermieden.}
    \item[\goodmark] Client reagiert schneller\\[2ex]
    \item[\badmark] Browser-Historie muss manuell befüllt werden
    \item[\badmark] keine Zugänglichkeit ohne JavaScript
      \only<article>{Suchmaschinen können nicht mehr so gut indizieren, spezielle
      Browser für Behinderte werden ausgeschlossen, \ldots}
    \end{itemize}
\end{frame}

\begin{Frame}{AJAX-Architektur}{}
  {\color<presentation>{maincolor}\bfseries Initialisierung}
  \begin{enumerate}
    \item Client vom Server laden 
      {\only<presentation>{\scriptsize\color{gray}}\newline
      über klassischen GET-Request im Browser}
    \item Client starten 
      {\only<presentation>{\scriptsize\color{gray}}\newline
      über eingebundenes JavaScript}
  \end{enumerate}    
  
  \only<presentation>{\vfill}
  
  {\color<presentation>{maincolor}\bfseries Interaktionsschleife}
  \begin{enumerate}
    \item Daten als JSON laden 
      {\only<presentation>{\scriptsize\color{gray}}\newline
      über GET-Request mit \texttt{XMLHttpRequest}-Objekt}
    \item Daten darstellen 
      {\only<presentation>{\scriptsize\color{gray}}\newline
      über Manipulation des Document Object Model (DOM)}
    \item auf Anwender-Interaktion reagieren 
      {\only<presentation>{\scriptsize\color{gray}}\newline
      über Ereignis-Registrierung im DOM}
    \item Daten an Server senden
      {\only<presentation>{\scriptsize\color{gray}}\newline
      über POST/PUT/DELETE-Request mit \texttt{XMLHttpRequest}}
  \end{enumerate}
\end{Frame}

\mode
<article>

AJAX stand ursprünglich für
\textbf{A}synchronous \textbf{J}avaScript \textbf{a}nd \textbf{X}ML.
Inzwischen wird meistens die JavaScript Object Notation (JSON)
verwendet und es sind auch synchrone Anfragen mölich, sodass
AJAX eher ein stehender Begriff für einen Architkturstil ist als
eine konkrete Beschreibung der verwendeten Technik.

\mode
<all>

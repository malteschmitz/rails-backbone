\newcommand{\authorstring}{Malte Schmitz, Johannes Thorn}
\newcommand{\shortauthorstring}{Malte, Johannes}
\newcommand{\titlestring}{Moderne Webanwendungen\\ mit Rails und Backbone.js}
\newcommand{\shorttitlestring}{Rails \&\only<presentation>{\\} Backbone}
\newcommand{\datestring}{MetaNook 2012}
\newcommand{\titleimage}{\pgfimage[height=12mm]{images/rails}{\huge\bfseries\raisebox{12pt}{\ \&\ }}\pgfimage[height=13mm]{images/backbone}}

\begin{document}

\displaytitle

\targets{
  \item Konzepte moderne Webapplikationen und Single-Page-Application mit JavaScript verstehen
  \item Prinzipien eines Webservice mit Rails 3.2 kennen
  \item Prinzipien eines JavaScript-Client mit Backbone kennen
  \item Verwendete Technologien und Frameworks gesehen haben
}

\begin{frame}{Github}
  \only<article>{Auf der Seite}
  \url{github.com/malteschmitz/rails-backbone}
  \only<article>{befinden sich}
  \begin{itemize}
    \item diese Präsentation
    \item Script-Version
    \item \LaTeX-Quellcode
  \end{itemize}
\end{frame}

\colorlet{maincolor}{nicegreen} % Color section in green
\section{Webanwendungen}

\begin{frame}{Webanwendungen vs. Desktopanwendung}
  \begin{center}
    \begin{tikzpicture}[
    box/.style={
      anchor=center,
      rotate=90,
      minimum width=#1,
      minimum height=5ex,
      inner sep=0pt,
      font=\small,
      draw=maincolor,
      thick,
      fill=maincolor!30,
      rounded corners
    },
    app/.style={
      anchor=center,
      rotate=90,
      inner sep=0pt,
      font=\large
    }
  ]
  
  \fill[maincolor!18]
    (0,5ex) rectangle (25em,15ex)
    (0,25ex) rectangle (25em,35ex);
  \fill[maincolor!9]
    (0,-5ex) rectangle (25em,5ex)
    (0,15ex) rectangle (25em,25ex);
  
  \begin{scope}[anchor=west,xshift=1em,font=\large]
    \node at (0,30ex) (Darstellung) {Darstellung};
    \node at (0,20ex) (Präsentation) {Präsentation};
    \node at (0,10ex) (Geschäftslogik) {Geschäftslogik};
    \node at (0,0) (Datenhalung) {Datenhaltung};
  \end{scope}
  
  \begin{scope}[xshift=10.5em]  
    \node at (0em,15ex) [app] {Desktop Application};
    \node at (2em,20ex) [box=28ex] {Client};
    \node at (2em,0ex) [box=8ex] {Server};
    
    \node at (5em,15ex) [app] {Web Application};
    \node at (7em,30ex) [box=8ex] {Client};
    \node at (7em,10ex) [box=28ex] {Server};
    
    \node at (10em,15ex) [app] {Single-Page-Application};
    \node at (12em,25ex) [box=18ex] {Client};
    \node at (12em,5ex) [box=18ex] {Server};
  \end{scope}  
\end{tikzpicture}
  \end{center}
\end{frame}

%%%%%%%%%%%%%%%%%%%%%%%%%%%%%%%%%%%%%%%%%%%%%%%%%%%%%%%%%%%%%%%%%%%%%%%%%%%%%%%%
\subsection{moderne Webanwendungen}

\begin{Frame}{moderne Webanwendungen}{}
  \begin{columns}
    \column{2cm}
    \column{\textwidth}
    {\color<presentation>{maincolor}\bfseries Geschäftslogik im Server} \newline
      \hspace*{1em} zentrale Berechnung und Datenhaltung
    
    \vskip1ex
    
    {\color<presentation>{maincolor}\bfseries zustandsloser Server} \newline
      \hspace*{1em} Abfragen in sich komplett
    
    \vskip1ex
    
    {\color<presentation>{maincolor}\bfseries Daten auf der Leitung} \newline
      \hspace*{1em} Daten frei von Darstellung
    
    \vskip1ex
      
    {\color<presentation>{maincolor}\bfseries Präsentation im Client} \newline
      \hspace*{1em} spezielle Clients möglich
  \end{columns}
\end{Frame}

%%%%%%%%%%%%%%%%%%%%%%%%%%%%%%%%%%%%%%%%%%%%%%%%%%%%%%%%%%%%%%%%%%%%%%%%%%%%%%%%
\subsection{RESTful Webservice}

\begin{Frame}{Was sind Webservices}
  \begin{itemize}
    \item Stellen Funktionalitäten über ein Netzwerk bereit
    \item Eindeutig definierte Schnittstelle
    \item Kommunikation über definierte Protokolle
    \item Zwei bekannte Vertreter: SOAP und RESTful
  \end{itemize}
\end{Frame}

\begin{Frame}{Representational State Transfer (REST)}
  \begin{tabularx}{\textwidth}{rX}
    \textbf{\color<presentation>{maincolor}Identität} & Ressource eindeutig identifiziert \\
    \textbf{\color<presentation>{maincolor}Manipulation} & Selbst beschreibende Ressourcen \\
    \textbf{\color<presentation>{maincolor}Nachrichten} & Alle Informationen 
      zur Verarbeitung \newline
      enthalten\\
    \textbf{\color<presentation>{maincolor}Hypermedia} & 
      \textcolor<presentation>{maincolor}{--} Wenige Einstiegspunkte \newline
      \textcolor<presentation>{maincolor}{--} Interaktion durch Hypermedia \\
    \textbf{\color<presentation>{maincolor}Zustandslos} & Status einer Interaktion mit dem
      Client unbekannt \\
  \end{tabularx}
\end{Frame}

\begin{Frame}{Was sind RESTful Webservices}
  \begin{itemize}
    \item Representational State Transfer (REST)
    \item Bekannter Vertreter WWW
    \item HTTP als Kommunikationsprotokoll
    \item Ressourcen als Elementen der Domäne
    \item Nutzt Verben des HTTP-Protokolls
    \begin{description}
      \item[GET] zum Informationsabruf
      \item[POST] zum Anlegen neuer Daten
      \item[PUT] zum Aktualisieren 
      \item[DELETE] zum Löschen von Daten
    \end{description}
  \end{itemize}
\end{Frame}

%%%%%%%%%%%%%%%%%%%%%%%%%%%%%%%%%%%%%%%%%%%%%%%%%%%%%%%%%%%%%%%%%%%%%%%%%%%%%%%%
\subsection{Single-Page-Application}

\begin{frame}{Single-Page-Application}
  \begin{itemize}
    \item gesamter Client durch Laden einer Website im Browser
    \item Daten werden per AJAX bei Bedarf nachgeladen\\[2ex]
    \item[\goodmark] Client agiert als Anwendung\only<article>{:
      Navigationselemente werden nicht neu geladen und Flackern der Seite wird
      vermieden.}
    \item[\goodmark] Client reagiert schneller\\[2ex]
    \item[\badmark] Browser-Historie muss manuell befüllt werden
    \item[\badmark] keine Zugänglichkeit ohne JavaScript
      \only<article>{Suchmaschinen können nicht mehr so gut indizieren, spezielle
      Browser für Behinderte werden ausgeschlossen, \ldots}
    \end{itemize}
\end{frame}

\begin{Frame}{AJAX-Architektur}{}
  {\color<presentation>{maincolor}\bfseries Initialisierung}
  \begin{enumerate}
    \item Client vom Server laden 
      {\only<presentation>{\scriptsize\color{gray}}\newline
      über klassischen GET-Request im Browser}
    \item Client starten 
      {\only<presentation>{\scriptsize\color{gray}}\newline
      über eingebundenes JavaScript}
  \end{enumerate}    
  
  \only<presentation>{\vfill}
  
  {\color<presentation>{maincolor}\bfseries Interaktionsschleife}
  \begin{enumerate}
    \item Daten als JSON laden 
      {\only<presentation>{\scriptsize\color{gray}}\newline
      über GET-Request mit \texttt{XMLHttpRequest}-Objekt}
    \item Daten darstellen 
      {\only<presentation>{\scriptsize\color{gray}}\newline
      über Manipulation des Document Object Model (DOM)}
    \item auf Anwender-Interaktion reagieren 
      {\only<presentation>{\scriptsize\color{gray}}\newline
      über Ereignis-Registrierung im DOM}
    \item Daten an Server senden
      {\only<presentation>{\scriptsize\color{gray}}\newline
      über POST/PUT/DELETE-Request mit \texttt{XMLHttpRequest}}
  \end{enumerate}
\end{Frame}

\mode
<article>

AJAX stand ursprünglich für
\textbf{A}synchronous \textbf{J}avaScript \textbf{a}nd \textbf{X}ML.
Inzwischen wird meistens die JavaScript Object Notation (JSON)
verwendet und es sind auch synchrone Anfragen mölich, sodass
AJAX eher ein stehender Begriff für einen Architkturstil ist als
eine konkrete Beschreibung der verwendeten Technik.

\mode
<all>

%\begin{frame}{}
%  \begin{itemize}
%    \item item
%  \end{itemize}
%\end{frame}

\colorlet{maincolor}{bordeaux} % Color section in bordeaux

\section[Webservice]{Webservice mit Rails 3.2}

\begin{frame}{Was ist Rails}
  \begin{itemize}
    \item Ruby on Rails Web-Framework
    \item Annahmen über \enquote{The Rails Way}
    \item Folgt Prinzipien
    \begin{itemize}
      \item DRY -- \enquote{Don’t Repeat Yourself}
      \item Convention Over Configuration
      \item REST-Architekturstil
    \end{itemize}
  \end{itemize}
\end{frame}

%%%%%%%%%%%%%%%%%%%%%%%%%%%%%%%%%%%%%%%%%%%%%%%%%%%%%%%%%%%%%%%%%%%%%%%%%%%%%%%%
\subsection{Model-View-Controller}

\begin{frame}{Models}
  \begin{itemize}
    \item Repräsentieren Daten
    \item Interagieren mit der Datenbank
    \item Enthalten Applikationslogik
    \item Häufig: Model $\hat =$ Datenbank-Tabelle
  \end{itemize}
\end{frame}

\begin{frame}{Views}
  \begin{itemize}
    \item Bieten Benutzungsschnittstelle der Applikation
    \item Präsentieren ausschließlich Daten
    \item Liefern Daten aus
    \item Häufig: HTML-Dokumente mit eingebettetem Ruby-Code
  \end{itemize}
\end{frame}

\begin{frame}{Controllers}
  \begin{itemize}
    \item Integrieren Views und Models
    \item Verarbeiten Anfragen
    \item Rufen Daten beim Model ab
    \item Liefern Daten zum View
  \end{itemize}
\end{frame}

%%%%%%%%%%%%%%%%%%%%%%%%%%%%%%%%%%%%%%%%%%%%%%%%%%%%%%%%%%%%%%%%%%%%%%%%%%%%%%%%
\subsection{REST mit Rails}

\frame{no content}

%%%%%%%%%%%%%%%%%%%%%%%%%%%%%%%%%%%%%%%%%%%%%%%%%%%%%%%%%%%%%%%%%%%%%%%%%%%%%%%%
\subsection{Responder aus Rails 3}

\frame{no content}

%%%%%%%%%%%%%%%%%%%%%%%%%%%%%%%%%%%%%%%%%%%%%%%%%%%%%%%%%%%%%%%%%%%%%%%%%%%%%%%%
\subsection{Views}

\frame{no content}

%%%%%%%%%%%%%%%%%%%%%%%%%%%%%%%%%%%%%%%%%%%%%%%%%%%%%%%%%%%%%%%%%%%%%%%%%%%%%%%%
\subsection{Assoziationen}

\frame{no content}

%%%%%%%%%%%%%%%%%%%%%%%%%%%%%%%%%%%%%%%%%%%%%%%%%%%%%%%%%%%%%%%%%%%%%%%%%%%%%%%%
\subsection{Validierung}

\frame{no content}



\colorlet{maincolor}{niceblue} % Color section in blue
\section[Client]{Client mit Backbone.js}

\begin{frame}{Architektur}
  \begin{center}
    \only<presentation| presentation:1>{
  \tikzset{
    every label/.style={
      font=\scriptsize\color{white}\bfseries,,
      inner sep=2pt
    }
  }
}
\only<2>{
  \tikzset{
    every label/.style={
      font=\scriptsize\color{maincolor}\bfseries,
      inner sep=2pt
    },
  }
}

\begin{tikzpicture}[
    node/.style={
      draw=maincolor,
      thick,
      rectangle,
      rounded corners,
      minimum width=6em,
      minimum height=4ex,
      inner sep=0ex,
      fill=maincolor!18},
    shorten >=1pt,
    line/.style={
      thick,
      color=maincolor
    },
    xs/.style={every edge/.append style={
      transform canvas={xshift=#1}
    }},
    ys/.style={every edge/.append style={
      transform canvas={yshift=#1}
    }},
    node distance=2em and 5ex
  ]

  \node[node, label={[name=Router label]BACKBONE}] (Router) {Router};
  \node[node, below=of Router, label={[name=StateModel label]CUSTOM}] (StateModel) {StateModel};
  \node[node, node distance=3em and -5ex, below right=of StateModel,label={[name=Views label]BACKBONE}] (Views) {Views};
  \node[node, node distance=3em and -5ex, above right=of Views,label={[name=DOM label]JQUERY}] (DOM) {DOM};
  \node[inner sep=0pt,above=of DOM] (Browser) {\pgfimage[width=4em]{images/browser}};
  \node[node, right=of Views,label={[name=Templates label]UNDERSCORE}] (Templates) {Templates};
  \node[node, below=of Views,label={[name=Collections label]BACKBONE}] (Collections) {Collections};
  \node[node, right=of Collections, label={[name=Ajax label]JQUERY}] (Ajax) {Ajax};
  \node[right=of Ajax,text width=4em,align=center] (Server) {\pgfimage[width=4em]{images/server}\newline Server};
  \node[node, below=of Collections,label={[name=Models label]BACKBONE}] (Models) {Models};

  \draw[line]
    (Browser) edge[->,out=180,in=0] (Router);
  \only<presentation| presentation:1>{
    \draw[line]
      (Router) edge[->] (StateModel);
  }
  \only<2>{
    \draw[line]
      (Router) edge[->] (StateModel label);
  }


  \only<presentation| presentation:1>{
    \draw[line, xs=-1em]
      ($(StateModel.south) + (1em,0)$) edge[->,out=270,in=90] (Views)
      (Views) edge[->] (Collections)
      (Collections) edge[->] (Models);
  }
  \only<2>{
    \draw[line, xs=-1em]
      ($(StateModel.south) + (1em,0)$) edge[->,out=270,in=90] (Views label)
      (Views) edge[->] (Collections label)
      (Collections) edge[->] (Models label);
  }

  \only<presentation| presentation:1>{
    \draw[line, xs=1em]
      (Models) edge[->] (Collections)
      (Collections) edge[->] (Views)
      (Views) edge[->,out=90,in=270] ($(DOM.south) + (-1em,0)$);
  }
  \only<2>{
    \draw[line, xs=1em]
      (Models label) edge[->] (Collections)
      (Collections label) edge[->] (Views)
      (Views label) edge[->,out=90,in=270] ($(DOM.south) + (-1em,0)$);
  }

  \only<presentation| presentation:1>{
    \draw[line]
      (DOM) edge[->] (Browser);
  }
  \only<2>{
    \draw[line]
      (DOM label) edge[->] (Browser);
  }

  \draw[line, ys=1ex]
    (Views) edge[->] (Templates)
    (Collections) edge[->] (Ajax)
    (Ajax) edge[->] (Server);

  \draw[line, ys=-1ex]
    (Server) edge[->] (Ajax)
    (Ajax) edge[->] (Collections)
    (Templates) edge[->] (Views);

\end{tikzpicture}

\endinput


  \draw[
      ->,
      thick,
      color=maincolor,
      every edge/.append style={
        transform canvas={xshift=-1em}
      }
    ]
    (client) edge (webrick)
    (webrick) edge (routing)
    (routing) edge (controllers)
    (controllers) edge (models)
    (models) edge (database);

  \draw[
      ->,
      thick,
      color=maincolor,
      every edge/.append style={transform canvas={xshift=1em}}
    ]
    (database) edge (models)
    (models) edge (controllers)
    (webrick) edge (client);

  \draw[
      thick,
      color=maincolor
    ]
    (controllers.east) edge[->,out=0,in=-90] (views.south)
    (views.north) edge[->,out=90,in=0] (webrick.east);

  \draw[
      ->,
      thick,
      color=maincolor,
      every edge/.append style={transform canvas={yshift=.6ex}}
    ]
    (webrick) edge (files);

  \draw[
      ->,
      thick,
      color=maincolor,
      every edge/.append style={transform canvas={yshift=-.6ex}}
    ]
    (files) edge (webrick);
\end{tikzpicture}


  \end{center}
\end{frame}

%%%%%%%%%%%%%%%%%%%%%%%%%%%%%%%%%%%%%%%%%%%%%%%%%%%%%%%%%%%%%%%%%%%%%%%%%%%%%%%%
\subsection{JavaScript}

\frame{no content}

%%%%%%%%%%%%%%%%%%%%%%%%%%%%%%%%%%%%%%%%%%%%%%%%%%%%%%%%%%%%%%%%%%%%%%%%%%%%%%%%
\subsection{DOM-Manipulation}

\frame{no content}

%%%%%%%%%%%%%%%%%%%%%%%%%%%%%%%%%%%%%%%%%%%%%%%%%%%%%%%%%%%%%%%%%%%%%%%%%%%%%%%%
\subsection{REST-Client}

\frame{no content}

%%%%%%%%%%%%%%%%%%%%%%%%%%%%%%%%%%%%%%%%%%%%%%%%%%%%%%%%%%%%%%%%%%%%%%%%%%%%%%%%
\subsection{Browser-Historie}

\frame{no content}

%%%%%%%%%%%%%%%%%%%%%%%%%%%%%%%%%%%%%%%%%%%%%%%%%%%%%%%%%%%%%%%%%%%%%%%%%%%%%%%%
\subsection{StateModel}

\frame{no content}

%%%%%%%%%%%%%%%%%%%%%%%%%%%%%%%%%%%%%%%%%%%%%%%%%%%%%%%%%%%%%%%%%%%%%%%%%%%%%%%%
\subsection{FormBuilder}

\frame{no content}

%%%%%%%%%%%%%%%%%%%%%%%%%%%%%%%%%%%%%%%%%%%%%%%%%%%%%%%%%%%%%%%%%%%%%%%%%%%%%%%%
\subsection{Abhängigkeiten}

\frame{no content}



\colorlet{maincolor}{niceviolet} % Color this section back in green
\section*{Zusammenfassung}

\begin{frame}{Zusammenfassung}
  \begin{enumerate}
    \item \alert{Moderne Webanwendungen} kombinieren \alert{Vorteile von
      Webanwendungen und Desktopanwendungen}.
    \item \alert{REST als Schnittstelle} erlaubt starke
      \alert{Entkopplung von Client und Server}.
    \item \alert{Rails} wird durch Einsatz von \alert{Respondern} und
      \alert{Representative View} zu einem mächtigen Server
      für \alert{RESTful Webservices}.
    \item Mit \alert{Backbone.js} stehen \alert{Modelle} des Servers dem
      Client zur Verfügung und können \alert{direkt manipuliert} werden.
    \item Die vorgestellte Lösung ist \alert{technologisch anspruchsvoll}, bietet
      aber viele \alert{Vorteile gegenüber klassischen Web-Frameworks}.
  \end{enumerate}
\end{frame}

\mode
<all>

\end{document}

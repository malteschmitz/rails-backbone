\begin{tikzpicture}[
    box/.style={
      anchor=center,
      rotate=90,
      minimum width=#1,
      minimum height=5ex,
      inner sep=0pt,
      font=\small,
      draw=maincolor,
      thick,
      fill=maincolor!30,
      rounded corners
    },
    app/.style={
      anchor=center,
      rotate=90,
      inner sep=0pt,
      font=\large
    }
  ]

  \fill[maincolor!18]
    (0,5ex) rectangle (25em,15ex)
    (0,25ex) rectangle (25em,35ex);
  \fill[maincolor!9]
    (0,-5ex) rectangle (25em,5ex)
    (0,15ex) rectangle (25em,25ex);

  \begin{scope}[anchor=west,xshift=1em,font=\large]
    \node at (0,30ex) (Darstellung) {Darstellung};
    \node at (0,20ex) (Präsentation) {Präsentation};
    \node at (0,10ex) (Geschäftslogik) {Geschäftslogik};
    \node at (0,0) (Datenhalung) {Datenhaltung};
  \end{scope}

  \begin{scope}[xshift=10.5em]
    \node at (0em,15ex) [app] {Desktop Application};
    \node at (2em,20ex) [box=28ex] {Client};
    \node at (2em,0ex) [box=8ex] {Server};

    \node at (5em,15ex) [app] {Web Application};
    \node at (7em,30ex) [box=8ex] {Client};
    \node at (7em,10ex) [box=28ex] {Server};

    \node at (10em,15ex) [app] {Single-Page-Application};
    \node at (12em,25ex) [box=18ex] {Client};
    \node at (12em,5ex) [box=18ex] {Server};
  \end{scope}
\end{tikzpicture}

